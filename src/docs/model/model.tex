\documentclass[a4paper]{article}

\title{Trajectory Model Code}
\author{Jamie Kettleborough}

\begin{document}
\maketitle

\section{Introduction}

This document outlines the numerical method and code used by the BADC
trajectory model.  It does not outline the details of the CGI
interface.
These are given in separate document.

\section{Numerical methods}

\subsection{Forward Time Stepping}
The time stepping of the trajectory positions is performed by a fourth
order Runge Kutta method.  Details of the Runge Kutta method can be
found in standard texts such as `Numerical Recipes'.  Stepping the
trajectories forward in time is simply solving the equation
\begin{displaymath}
 \frac{d\bf{x}}{dt}=\bf{u}(\bf{x},t)
\end{displaymath}
where $\bf{x}$ are the parcel positions, $\bf{u}$ are the parcel
velocities, and $t$ is time.

The method requires interpolation of the forcing wind
fields onto the parcel positions.  The Runge Kutta requires the
velocities at three time levels during each time step: the current
time ($t$), halfway through the time step ($t+\frac{\delta t}{2}$), and
at the end of the time step ($t+\delta t$).  Depending on the relative
number of parcels being advected and the number of grid points on the
wind forcing fields it can be more efficient to interpolate in time
first or in space first.  When interpolating in time first
interpolation has to be done at all the forcing field grid points.
This can be more efficient, however, since the wind from the end of
one time step ($t+\delta t$) can be carried over to become the gridded
wind at the beginning of the next time step ($t$).  The code, as
written, supports both interpolation in time first or interpolation in
space first.  (In retrospect the way that this is supported in the
code is not very well implemented and complicates the code
unnecessarily.)

The main time stepping routine is called {\tt runge\_kutta}.

Some important relevant variables are

\begin{tabular}{|l|l|}
\hline
Variable Name & Meaning \\
\hline
{\tt velt} & Gridded wind at current time step \\
{\tt velthdt} & Gridded wind at centre of current time step \\
{\tt veltdt} & Gridded wind at end of current time step \\
{\tt ltfirst} & Switch time interpolation first on or off \\
\hline
\end{tabular}

\subsection{Spherical Geometry}
The advection is done on a sphere (so the geometry of the surfaces of
constant height variable are approximated by spheres).  This is
achieved by using Cartesian coordinates $(x,y,z)$ on a sphere.  The
local vertical coordinate (pressure, or potential temperature) is
supported by a fourth dimension ($zhm$).  The gridded winds are given
on a latitude-longitude grid and with eastward and northward
components.  The Cartesian parcel positions have to be transformed
into latitude-longitude coordinates, and the winds then transformed
into Cartesian components. During the time stepping the Cartesian
coordinate of the parcels are renormalised to keep them on the surface
of a unit sphere.

Relevant routines include

\begin{tabular}{|l|p{15pc}|}
\hline
Routine & Purpose \\
\hline
{\tt latlon\_from\_cartesians} & convert Cartesian parcel positions in
lat-lon coordinates \\
{\tt find\_parcel\_cartesians} & convert lat-lon coordinates of parcels
to Cartesian \\
{\tt uvw} & part of this routine converts Eastward/Northward winds to
Cartesian coordinates\\
{\tt adjust} & remornamlise the Cartesian parcel positions to lie on a
unit sphere \\
\hline
\end{tabular}

\subsection{Time interpolation}
Simple linear interpolation is used to get the velocities onto the
current time.  The routine {\tt time\_interp} performs the time
interpolation.

\subsection{Space interpolation}
The space interpolation is also performed by linear interpolation.
The routine {\tt uvw} is responsible for the interpolation (as well as
transforming the interpolated winds into Cartesian components).  {\tt
  uvw} calls helper routines:

\begin{tabular}{|l|p{15pc}|}
\hline
Routine & Purpose \\
\hline
{\tt get\_horizontal\_weights } & Calculate the interpolation weights
for the horizontal direction \\
{\tt find\_vert} & find the location of the parcels on the vertical
grid and calculate vertical interpolation weight\\
{\tt interpolate} & built in IDL routine for linear interpolation\\
\hline
\end{tabular}

\subsection{Vertical Coordinates}
The trajectory model supports parcel vertical coordinates of hybrid
pressure or potential temperature.  When the vertical
coordinate of the parcel positions is the same as that of the wind
field then the wind grid information simply carries the one
dimensional level information.  If the vertical coordinate of the
parcel positions is not the same as those of the gridded wind field
then there has be be a way of calculating the the gridded values of
the parcel vertical coordinate from the gridded fields, this gridded
vertical coordinate is then carried around as the three dimensional
level information.  This has to be done, for instance, when the parcel
vertical coordinate is potential temperate, but the gridded wind
vertical coordinate is pressure, or hybrid sigma-pressure.

\subsection{Undefined points}
Parcels that are initialised close to the boundaries of the wind
domain may be advected out of the domain.  These parcels are set as
undefined.  Within the code a list of undefined parcels is carried
around as the variable {\tt in\_range}.  Only parcels that are defined
are advected.  Carrying the {\tt in\_range} array around probably is
not the best way to account of for undefined parcels.  Simply setting
the coordinates as $-999$ may be better.

\section{Outline}
An outline of the steps made by the code is as follows:

\begin{enumerate}
\item initialise parcel positions (external to main model)
\item set inputs
\item initialise model grid (defines grid wind files are present on)
\item read first wind file
\item loop around time
\begin{enumerate}
\item read and swap winds if needed
\item new release if needed (include interpolation of temperature etc)
\item perform Runge-Kutta step
\item increment time step
\item output on this time step if needed.
\item delete an old release if needed.
\end{enumerate}
\item close any unclosed output files
\end{enumerate}

\section{Multiple Releases}
Many users want to run the same trajectories for a number of different
times, say to build up a climatology of air origins at their observing
station.  To help with this kind of study trajectories can be run as a
set of multiple release trajectories.  In this case trajectories are
run with identical spatial starting positions for a set of times.

\section{Time step and date handling}
The trajectory model uses a set of routines, such as {\tt dt\_add}
that  were available at IDL 5.2, but which have since been
removed.  They have to be made available (in Traj:Job.pm) in order to
run the trajectory model.

\end{document}