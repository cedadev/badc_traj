\documentclass[a4paper]{article}

\usepackage{epsfig}

\title{Trajectory Model Interface}
\author{Jamie Kettleorough}

\oddsidemargin -0.5in
\evensidemargin -0.5in
\topmargin -0.5in
\textwidth 7.2in
\textheight 10.in

\begin{document}
\maketitle

\section{Introduction}

Trajectory model is simple parcel advection written at BADC.  The
parcel advection is performed by IDL.  The WWW based interface is
written in Perl.  This doc deals with the interface.

\section{setuid}

The interface is run under the WWW server which will usually be run as
a user with low privileges (e.g. webuser).  The trajectory service has
to access files owned by badc and with different group permissions.
To enable this the WWW based forms have to run as setuid/setgid
programs.  IDL licensing detects when the real and effective user id's
are different, so have to setuid root and then change both to badc.

There is a race condition which means that Perl scripts cannot be run
as setuid and so they all have to be wrappered in a c program.  This
means all form processing CGI that requires setuid have three files
associated with them.

\begin{itemize}
\item Perl source
\item c wrapper source
\item executable
\end{itemize}
(In what follows the Perl source files will be used to refer to the
setuid CGI form processing).

Setuid bits are as follows
\begin{verbatim}
-rwsr-sr-x    1 root     badcint     11709 Nov  7  2003 find_file
-rwsr-sr-x    1 root     badcint     11803 Nov  7  2003 nc2nasa
-rwsr-sr-x    1 root     badcint     11802 Nov  7  2003 nc2www
-rwsr-sr-x    1 root     byacl       11696 Nov  7  2003 plot
-rwsr-sr-x    1 root     ecmwftrj    11709 Nov  7  2003 plot_traj
-rwsr-sr-x    1 root     badcint     11709 Nov  7  2003 traj_form
\end{verbatim}
(Note these setuid bits are the working version setuid bits as of 8th
August 2005.  They will make the trajectory model work, but are not
necessarily the only, or best, values to use.)

The setuid bits can be set using chmod and chown when logged on as
root.

\section{Outline of calling}

There are two entry points for the trajectory model

\begin{itemize}
\item setting up a trajectory job using traj\_form.p
\item plotting a trajectory output file using plot\_traj.p
\end{itemize}

To browse and download files produced by the trajectory service the
BADC databrowser is used (though in earlier version there was a
trajectory specific file browser, called find\_file.p).

\subsection{traj\_form.p}

The CGI script traj\_form.p produces three HTML pages depending on
how it has been called.

\begin{enumerate}
\item the trajectory submission form for completion by user to define the
  trajectory option they want to run with (start date etc.).  This
  form is produced on the first call to traj\_form.p, and whenever
  there are errors in user input that need correcting, or when input
  data files are missing.
\item the trajectory confirm form that details the trajectory run the user
  has defined and gives the user a chance to change them.  This form
  is produced when a trajectory run has been fully and correctly
  defined by the user and all files needed are present.
\item the information page that says a trajectory job has been
  submitted. This page is produced after the user has confirmed they
  would like to submit the trajectory run.
\end{enumerate}

In producing the pages some processing goes on behind the scenes, for
instance, to check consistent input, to check that files are present,
to log details of the trajectory request etc. On overview of the
processing for a successful call to the trajectory service is given in
figure \ref{fig:formsubmission}.
  Details of the checks are best seen through the pod to the
Check.pm module.  The checks rely on generating filenames for the
dataset used to generate the trajectories.  This filename generation
is done by a series of classes, each class provides functions giving
information about each dataset.  The interface is given in figure 
\ref{fig:filetype},
descriptions of each function can be found in the pod of the
File\_type.pm module.

\begin{figure}
\begin{center}
\epsfxsize=40pc \epsfbox{formsubmission.eps}
\end{center}
\caption{Successful user submission of a trajectory job}
\label{fig:formsubmission}
\end{figure}

\begin{figure}
\begin{center}
\epsfxsize=30pc \epsfbox{filetype.eps}
\end{center}
\caption{File type interface}
\label{fig:filetype}
\end{figure}

The submission of the job is to a queue.  This prevents blocking of
the browser while the trajectories are calculated.  The queuing system
used is generic NQS, unfortunately it looks as though this is no
longer distributed or supported.  One consequence of this is that
there is no direct submission of jobs from foehn (the web server
machine) to mercury (where the trajectories are run).  Jobs are
submitted to the queue by a ssh from foehn to mercury.  This is not
great since it adds another point of failure.

After submission to the queue the user is informed of progress via
e-mail.  The mails are generated by the job created by
Job::job\_submit.  Failed jobs also generate an e-mail that goes to
badc support to be forwarded to the trajectory model maintainer.

\subsection{plot\_traj.p}

The CGI script plot\_traj.p deals with producing the form to be
completed by the user to specify what plots are needed.  It produces
two variants of basically the same page.  The first time plot\_traj.p
is called it simply gives the selection boxes to specify the plot.
On subsequent calls to plot\_traj the actual image asked for will be
embedded in the form (provided there are no errors).  The image is
embedded using a HTML $<$IMG$>$ tag where the image is another CGI script
called plot.idl.  In practice plot.idl has to be wrappered by plot.c
(and plot.pl) for the sake of setting the relevant environment and setuid.

plot\_traj.p can only deal with one file at a time.  Its main purpose
it to open the file to see what variables it contains.  The variables
are then used to determine what plot types are offered to the user.

plot.idl is a idl batch file which sets up the correct idl paths and
ensures that some needed procedures are compiled.
It then calls plot\_traj\_cgi.pro.  plot\_traj\_cgi.pro reads and
decodes the CGI environment (in idl) and passes the decoded form
parameters onto plot\_traj.pro.  plot\_traj.pro actually does the
plotting.  Other BADC idl routines are needed by the plotting.  These
decode the form parameters and are usually stored in the {\tt
cgi-bin/idl} directory under the WWW server.  This is quite convoluted
and takes a little getting used to.  The plotting is robust though and
so it is only necessary to understand all these processes when you
want to make changes.  Figure \ref{fig:plot_sequence} outlines the
relationships between the different programs.

\begin{figure}
\begin{center}
\epsfxsize=8pc \epsfbox{plot_sequence.eps}
\end{center}
\caption{Relationships between plotting programs, note plot\_traj\_cgi.pro and plot\_traj are in the subdirectory idl}
\label{fig:plot_sequence}
\end{figure}


(On retrospect this could be made simpler, at least, by having plot wrapper a
Perl cgi script that calls the IDL plot\_traj.pro, it would involve
fewer files, and the setuid wrapping would look more like the setuid
wrapping for traj\_form etc.)

For historic reasons the plot\_traj uses a specialized data browser if
the file path asked for does not exist.  This is called find\_file.p.

\section{Source files and output files}

The source files are found under the relevant dataset directories of
/badc.  The output files are written to
/requests/[username]/traj\_service.
If a trajectory job fails then the output is saved in a file in
/tmp/trajectory on mercury.  This can then be used to discover the
source of the error.

\section{Common reasons for fail}

Note these reasons have been taken from memory.  The symptoms may not
be exactly right.  The lists is a good list of things to check when
there are problems.

\begin{tabular}{|p{100pt}|p{100pt}|p{100pt}|}
\hline
{\bf Symptom} & {\bf Likely cause} & {\bf Check} \\
\hline
Fail e-mail & disks with input data have become unmounted on mercury & log file \\
\hline
Confirm form freezes & mercury down & ping mercury \\
\hline
Confirm form freezes & mercury keys ssh keys invalid for some reason & ssh badc@mercury from foehn \\
\hline
 & qsub broken on mercury & from mercury do qsub or qstat \\
\hline
Jobs appear to complete  but no mails & mail broken on mercury & test mail on mercury \\
\hline
 & /tmp/trajectory deleted on mercury & \\
\hline
Server error from first request of traj\_form & changes in core BADC Perl modules that have not been fed through to the trajectory code & talk to Andrew \\
\hline
\end{tabular}

\section{Suggested system fail warning system}

Some of the common errors could be avoided by running a set of checks
as, say, twice daily cron jobs that mail on error.  These could be put
under the cwrapper program to make consistent with other BADC cron
jobs that mail information about stderr and stdout

\begin{enumerate}
\item check disks are mounted on mercury
\item ssh to mercury as badc from foehn
\item qsub a simple job on mercury (asynchronous, so may be harder?)
\end{enumerate}

\section{Suggested Improvements}

\begin{enumerate}
\item make it easier to add new datasets at the interface level. (replace sub routine source\_class with a hash)
\item replace NQS with a supported queuing system.
\item remove plot\_traj use of find\_file and retire find\_file.
\item replace plot\_traj with a GEOSPLAT version that supports
trajectories.
\end{enumerate}

\end{document}